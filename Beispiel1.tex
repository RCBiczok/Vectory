\documentclass[a4paper,11pt,german]{article}
\usepackage{graphicx}
\usepackage{amsmath}
\usepackage[latin1]{inputenc}
\usepackage {layout}
\usepackage{ngerman}
\usepackage{color}


% R�nder setzen
\setlength{\voffset}{-5.4mm}

\setlength{\hoffset}{-5.4mm}

\setlength{\topmargin}{-0.5cm}

\setlength{\oddsidemargin}{1.5cm}

\setlength{\textheight}{23.2cm}

\setlength{\textwidth}{15cm}

\makeatletter

\begin{document}

\textbf{1.Beispiel}

Das erste Beispiel zeigt, wie man mit Vectory eine umfangreiche
Aufgabe darstellen und ohne selbst zu rechnen l�sen kann.

Aufgabe 34, Buch S.264:
\[
g_a:\overrightarrow{X}=
\begin{pmatrix}
  2a \\
  a+1 \\
  1 \\
\end{pmatrix}+
\mu
\begin{pmatrix}
  0 \\
  1 \\
  3 \\
\end{pmatrix},\;E: 3x_1-6x_2+2x_3+4=0,\;A(15|1|3)
\]
\renewcommand{\theenumi}{\textbf{\alph{enumi})}}
\begin{enumerate}
    \item Untersuche die Lage von $g_a$ und E.
    \item Welcher Punkt B in E liegt A am n�chsten?
    \item Welche Schargerade liegt A am n�chsten?
    \item E sei Tangentialebene einer Kugel k um A. Berechne Radius und Ber�hrpunkt.
    \item Bestimme eine Gleichung der Ebene H, die die Kugel k von
    \textbf{d)} halbiert und auf den Schargeraden senkrecht steht.
    \item Bestimme eine Gleichung der Schnittebene s von H und E.
\end{enumerate}

\end{document}
